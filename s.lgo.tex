\hfill{\Large\bf cmput 396 ~ solving linear go ~ prof hayward}\hfill~

\section*{linear go}
We discuss briefly
the problem of solving go on small boards.
We consider
{\bf linear go}, ie.\ go on a 1$\times$$n$ board, for some $n$.
We will consider the simplest go rule set:
Tromp-Taylor (so, positional superko)
with no suicide (so, each move is either pass,
or a play that leaves the stone played in a group
that has at least one liberty).

Anyone can solve any go position, given enough time and space:
just build the complete minimax search tree.
But, as you can see from the examples below, the
game trees can get large, so our goal is
to use efficient algorithms and
{\bf to prune child moves that do not change the 
parent's minimax value}.

Here is the game tree for 1x2 go with, 
at each node, isomorphic children pruned.

\[\begin{tikzpicture}[x=1cm, y=1cm
  ,every edge/.style={draw, postaction={decorate,decoration={markings,}}}
]
\vertex (a) at (5,9) {\pic{--}};
\vertex (b) at (3.5,8) { };
\vertex (c) at (6.5,8) {\pic{x-}};
\vertex (d) at (3,7) { };
\vertex (e) at (4,7) {\pic{o-}};
\vertex (f) at (6,7) { };
\vertex (g) at (7,7) {\pic{-o}};
\vertex (h) at (3.5,6) { };
\vertex (i) at (4.5,6) {\pic{-x}};
\vertex (j) at (6,6) {};
\vertex (k) at (7,6) {};
\vertex (l) at (3.5,5) {};
\vertex (m) at (4.5,5) {};
\vertex (n) at (7,5) {};
\vertex (o) at (4.5,4) {};
\path
(a) edge (b) edge (c)
(b) edge (d) edge (e)
(c) edge (f) edge (g)
(e) edge (h) edge (i)
(f) edge (j)
(g) edge (k)
(h) edge (l)
(i) edge (m)
(k) edge (n)
(m) edge (o)
;
\end{tikzpicture}\]

Here is the game tree (top 3 levels only)
for 1x3 go with, at each node, isomorphic children pruned.
On the diagram, draw the next level of the tree.
Then, on the diagram,
label each node with the first-player minimax score.

\[\begin{tikzpicture}[x=1cm, y=2cm
  ,every edge/.style={draw, postaction={decorate,decoration={markings,}}}
]
\vertex (a) at (5,9) {\pic{---}};
\vertex (b) at (2,8) { };
\vertex (c) at (5,8) {\pic{x--}};
\vertex (d) at (8,8) {\pic{-x-}};
\vertex (e) at (1,7) { };
\vertex (f) at (2,7) {\pic{o--}};
\vertex (g) at (3,7) {\pic{-o-}};
\vertex (h) at (4,7) { };
\vertex (i) at (5,7) {\pic{xx-}};
\vertex (j) at (6,7) {\pic{x-x}};
\vertex (k) at (8,7) { };
\path
(a) edge (b) edge (c) edge (d)
(b) edge (e) edge (f) edge (g)
(c) edge (h) edge (i) edge (j)
(d) edge (k)
;
\end{tikzpicture}\]
\vfill~
