\section*{a pruning theorem}
Define a go state $S=(P,x)$ by the history $P = (P_0, \ldots, P_t)$
of board positions $P_j$, 
where $P_0$ and $P_t$ 
are respectively the original and current position,
and $|P| = t+1$ is the number of positions in the history;
and by the player to move (ptm) $x$.

We label the cells of the linear go board from 0 to $n-1$
in the obvious way, ie.\ so that,
for $1 \leq j \leq n-1$, cells $j-1$ and $j$ are neighbours.

For a go state $S$, 
$\mu_x(S)$ is the minimax value for $x$ of $S$.

\begin{theorem}
Consider a linear go position $S=(H,P,x)$
with cells 1,2 both empty for all positions in the history.
Let $S_1$ (respectively $S_2$)
be the state obtained from $S$ after $x$ plays at cell 1 (cell 2).
Then $\mu_x(S_1) \leq \mu)x(S_2)$.
\end{theorem}

{\bf corollary}

In solving $S$, we can prune $S_1$.

~

Schematically we have $S=$ {\tt --"}, 
$S_1=$ {\tt x-"}, 
$S_2=$ {\tt -x"},
where {\tt -} represents an empty cell and {\tt "} represents cells 3\ldots$n$.

~

{\bf sketch of proof}

In the game tree, the children of $S_1$ are
{\tt -o"}, and (for every legal subposition {\tt '} of {\tt "} obtained
by a legal x-move) {\tt x-'}.
The children of $S_2$ are
(possibly, if the capture works)
{\tt o-"}, and {\tt -x'}.

We argue by induction 
(we need to reformulate the statement of the theorem so
that it holds for any states reached from the original
by a parallel sequence of moves, ie.\ that differ only
by interchanging cells 1 and 2)
on {\tt "} that
$\mu_o(${\tt o-"}$) \leq \mu_o(${\tt -o"}), so
$\mu_x(${\tt -o"}$) \leq \mu_x(${\tt o-"}).
$\mu_x(${\tt x-'}$) \leq \mu_x(${\tt -x'}).
This, and the fact that {\tt -o"} exists (even though {\tt o-"} exists
only if the capture is legal)
implies that $\mu_x(${\tt x-"}$=S_1) \leq \mu_x(${\tt -x"}$=S_2)$.
\vfill~
