\section*{a pruning conjecture}
Here, we discuss go assuming no-suicide Tromp-Taylor rules,
so with the positional superko rule.

So, define the {\em history} of a go state as the sequence
$\mP = (P_0, \ldots, P_{t-1})$
of board positions $P_j$, 
where $P_0$ and $P_{t-1}$ 
are respectively the original and current position,
and $|\mP| = t$ is the number of positions in the history.
Define a {\em go state} $S=(\mP,x)$ by its history $\mP$
and by the player to move (ptm) $x$.

Label the cells of the linear go board from 0 to $n-1$
in the obvious way, ie.\ so that,
for $1 \leq j \leq n-1$, cells $j-1$ and $j$ are neighbours.

For a go state $S$, 
$\mu_x(S)$ is the minimax value for $x$ of $S$.

For cells $x,y$ of a position $P$,
the {\em x,y-interchange} $\phi(x,y,P)$ of $P$ is
the position $P'$ obtained from $P$ by 
interchanging the contents of cells $x$ and $y$.
For example,
let $P$ be the linear go position {\tt ( x - o o - ) }.
Then $\phi(0,1,P)$ is the position {\tt ( - x o o - ) }.

Similarly,
the {\em x,y-interchange} $\phi(x,y,\mP)$ of $\mP$
is the history $\mP'$ obtained from $S$ by interchanging
the contents of cells $x$ and $y$ in each position $P_j$ of $\mP$.

{\bf comments}
we would like a pruning conjecture like the one below,
but the problem is that capturing stones makes comparison
of associated state pairs (one from the inferior move strategy,
the other from the superior) messy ...

\begin{itemize}
\item
does the proof hold if we insist the 1st 3 cells (not just the 1st 2)
are empty?  not sure \ldots
problem is when the superior move (to cell 2) captures opponent stones
\item maybe the idea of the proof is not the best: now
I'm wondering whether we should compare the inferior move strat
to the strat where opponent replies to cell 2 and kills it,
and then player still wins ... several cases to consider,
but maybe this is the best approach
\item new conjecture: if x creates the following fresh prefix
state,
then x has a 2nd player strategy that preserves the prefix
($+$ indicates string repeated $k\geq 1$ times
\item
similar conjecture:
what about the following pattern? tricky, or maybe not true
( [ ] indicates 0 or 1 time )
\end{itemize}

\begin{verbatim}
- { { x }+ - }+ sigma

- { x }+ { [ - ] - { x }+ }* - sigma
\end{verbatim}

\begin{conjecture}
Let $S^0 =(\mP,z)$ be a go state in which the current position
$P_j$ has cell 0 $x$ and cell 1 empty.
Let $\mP' = \phi(0,1,\mP)$, and let
$S^1 = (\mP',z)$ be a valid go state, ie.\ the sequence
of positions in $S_1$ corresponds to a sequence of legal moves.
Then $\mu_x(S^0) \leq \mu_x(S^1)$.
\end{conjecture}

We will sometimes represent positions pictorially:
eg.\ ({\tt x - }$\rho$) is a position with cell 0 x, cell 1 empty,
and remaining cell sequence $\rho$.

~

\begin{proof}
The current positions of $S^0$ and $S^1$ are respectively
({\tt x - }$\rho$) and ({\tt - x }$\rho$).
We want to find the minimax value for x of $S^0$ and $S^1$,
so let the player-to-move z be o, the opponent of x. 

First assume that $S_0$ is a terminal state:
the last move was pass, as was the immediately previous move by x.
Notice that this implies $S_1$ is also terminal.
Here, the minimax value is just the Tromp-Taylor score:
if the first non-empty cell in $\rho$ is x, $\mu_x(S_0)=\mu_x(S_1)$;
if not, $\mu_x(S_0) = \mu_x(S_1) - 1$.
So the theorem holds.

Next assume that neither $S_0$ nor $S_1$ are terminal.
Let $L^0$ and $L^1$
be the set of legal moves for o from $S^0$ and $S^1$ respectively.
$\mP'$ is the 0,1-interchange of $\mP$,
so $L^0 - \{0,1\} = L^* = L^1 - \{0,1\}$.

From $S^0$,
the move by o to cell 1 captures the x in cell 0,
so 1 is in $L^0$ if and only if the resulting position
({\tt - - }$\rho'$) is not in $\mP$,
where $\rho' = \rho$ unless $\rho$ begins with a sequence
$\sigma$ of consecutive x's followed by either o or the end of the board,
in which case these x's are also captured when o plays at cell 1.

Now consider from $S^1$ the move by o to cell 0.
If $\rho' = \rho$, then $\rho$ does not begin with a sequence $\sigma$
as described above, and this move does not capture the x
at cell 1, so this move is not legal.
Also, if $\rho' \neq \rho$, then
$\rho$ does begin with $\sigma$,
and so the position after o captures the x's at cells \{1\} 
and the locations of $\sigma$ will be exactly the same,
except for interchanging cells 0 and 1,
as the position after o captures the x's by playing at cell 1 from $S^0$.
So, since $\mP'$ is the 0,1-interchange of $\mP$,
if this move by o from $S^1$ to cell 0 is legal,
the so is the move by 1 from $S^0$ to cell 1.

So there are 3 cases:
$L^0 = L^* = L_1$;
$L^0 = L^* \cup \{1\}, L_1 = L*$;
$L^0 = L^* \cup \{1\}, L_1 = L* \cup \{0\}$.

In each case, observe that for every cell $c$ in $L^*$,
the position $Q_0$ obtained from the o-move to c from $S^0$ 
is the 0,1-interchange of 
the position $Q_1$ obtained from the o-move to c from $S^1$.
Also, the hypotheses of the theorem apply to $Q_0$ and $Q_1$,
so by induction we can assume that
$\mu_x(Q_0) \leq \mu_x(Q_1)$.
In the first case, $\mu_x(S_j) = \min_c \{\mu_x(Q_j)\}$
for each $j=0,1$, so
\[ \mu_x(S_0) \: = \: 
\min_c \{\mu_x(Q_0)\} \: \leq \:
\min_c \{\mu_x(Q_1)\} \: = \: 
\mu(S_1) \: , \]
and we are done.
In the second case, $L^0$ contains all the moves of $L^1$,
and one more, which has minimax value say $r$, so
\[ \mu_x(S_0) \: = \: 
\min \{ \: \min_c \{\mu_x(Q_0)\} \; , r \: \} \:  \leq \: 
\min_c \{\mu_x(Q_1)\} \: = \: 
\mu(S_1) \: , \]
and again we are done.
In the third case,
let $V_1$ ($V_0$) be the position
obtained by an o-move from $S_0$ ($S_1$) to cell 1 (cell 0),
let $\mu_o(V_1)$, let $r_0 = \mu_o(V_0)$.
The hypotheses of the theorem apply after exchanging o for x,
so $\mu_o(V_0) \leq \mu_o(V_1)$ by induction.
Also, for any position $A$, $\mu_o(R) = - \mu_x(R)$,
so $r_1 = \mu_x(V_1) \leq \mu_x(V_0) = r_0$, and we have
\[ \mu_x(S_0) \: = \: 
\min \{ \: \min_c \{\mu_x(Q_0)\} \; , r_1 \: \} \: \leq \: 
\min \{ \: \min_c \{\mu_x(Q_1)\} \; , r_0 \: \} \: = \:
\mu(S_1) \: . \]
\end{proof}


\begin{corollary}\label{cor}
For some $k\geq 1$,
let $S=(\mP,x)$ be a linear go state
such that the set of empty cells to which
$x$ has a legal move includes both $0$ and $1$.
Then the minimax value for $x$ for the move to $1$
is at least as good as the minimax value for the move to 0.
\end{corollary}

\begin{theorem}
For $k\geq 1$,
let $S^0 =(\mP,z)$ be a go state in which the current position
$P_j$ has cells $\{0,\ldots,k-1\}$ $x$ and cell $k$ empty.
Let $\mP' = \phi(0,k,\mP)$, and let
$S^1 = (\mP',z)$ be a valid go state, ie.\ the sequence
of positions in $S_1$ corresponds to a sequence of legal moves.
Then $\mu_x(S^0) \leq \mu_x(S^1)$.
\end{theorem}

\begin{proof}
The two positions here correspond to the positions obtained by
inserting $k-1$ x's between cells 0 and 1 
of the two positions of the previous theorem.
The proof is the same.
\end{proof}

\begin{corollary}\label{cor2}
For some $k\geq 1$,
let $S=(\mP,x)$ be a linear go state
such that the current position has x at cells 1,\ldots, k,
and the set of empty cells to which
$x$ has a legal move includes both $0$ and $k+1$.
Then the minimax value for $x$ for the move to $k+1$
is at least as good as the minimax value for the move to 0.
\end{corollary}
\vfill~

We can use the corollaries to prune moves when solving states.
Let us solve the initial position of 1$\times$4 go.
Up to ismorphism symmetry, there are 3 initial moves to consider:
pass, cell 0, cell 1.

\[\begin{tikzpicture}[x=1cm, y=1.5cm
  ,every edge/.style={draw, postaction={decorate,decoration={markings,}}}
]
\vertex (a) at (5,9) {\pic{----}};
\vertex (b) at (3,8) { };
\vertex (c) at (5,8) {\pic{x---}};
\vertex (d) at (7,8) {\pic{-x--}};
\path
(a) edge (b) edge (c) edge (d)
;
\end{tikzpicture}\]

By Corollary~\ref{cor}, the move to 
{\tt (- x - -)} is at least as good
as the move to 
{\tt (x - - -)}, so we prune the latter move.
Now let us explore the next levels from {\tt (- x - -)}.

\[\begin{tikzpicture}[x=1cm, y=1.5cm
  ,every edge/.style={draw, postaction={decorate,decoration={markings,}}}
]
\vertex (a) at (5,9) {\pic{----}};
\vertex (b) at (3,8) { };
\vertex (d) at (7,8) {\pic{-x--}};
\vertex (e) at (5,7) { };
\vertex (f) at (7,7) {\pic{-xo-}};
\vertex (g) at (9,7) {\pic{-x-o}};
\path
(a) edge (b) edge (d)
(d) edge (e) edge (f) edge (g)
;
\end{tikzpicture}\]

Now we can prune {\tt (- x - o)}.
\[\begin{tikzpicture}[x=1cm, y=1.5cm
  ,every edge/.style={draw, postaction={decorate,decoration={markings,}}}
]
\vertex (a) at (5,9) {\pic{----}};
\vertex (b) at (3,8) { };
\vertex (d) at (7,8) {\pic{-x--}};
\vertex (e) at (5,7) { };
\vertex (f) at (7,7) {\pic{-xo-}};
\vertex (h) at (3.5,6) { };
\vertex (i) at (4.5,6) {\pic{xx--}};
\vertex (j) at (5.5,6) {\pic{-xx-}};
\vertex (k) at (6.5,6) {\pic{-x-x}};
%\vertex (l) at (8,6) {\pic{-x-x}};
\path
(a) edge (b) edge (d)
(d) edge (e) edge (f)
(e) edge (h) edge (i) edge (j) edge (k)
;
\end{tikzpicture}\]
By Corollary~\ref{cor2},
we can prune {\tt (x x - -)}.
By Corollary~\ref{cor},
we can prune {\tt (- x - x)}.
\[\begin{tikzpicture}[x=1cm, y=1.5cm
  ,every edge/.style={draw, postaction={decorate,decoration={markings,}}}
]
\vertex (a) at (5,9) {\pic{----}};
\vertex (d) at (5,8) {\pic{-x--}};
\vertex (e) at (3,7) { };
\vertex (f) at (7,7) {\pic{-xo-}};
\vertex (j) at (3,6) {\pic{-xx-}};
\vertex (l) at (7,6) {\pic{-x-x}};
\path
(a) edge (d)
(d) edge (e) edge (f)
(e) edge (j)
(f) edge (l)
;
\end{tikzpicture}\]
Here is the final proof tree:
for each x-turn, we show a best move;
for each o-turn, we show all non-pruned moves.
The minimax score for x is 4.

\section*{problem with proof}
proof becomes messy if good move captures

eg

\begin{verbatim}
- - o o o o x - sigma, 2 replies are

- x - - - - x - sigma     <- good move
x - o o o o x - sigma     <- bad move
\end{verbatim}

problem: opponent has more options after good move.

can refill captured area, we have no way to decide
what to do... 2 options:

a) pass in response to each move
into captured set  

b) if there is a unique response to 
an evetual move by o to cell 2, then just play that way

observation:
if we assume cells 1 and 2 have never been played,
then when the stones at 3,4,5,6 in this example were played,
it was for each the 1st time those cells were occupied
(because capture would require that cell 2 be played at)

observation:
for $3 \leq n \leq t$, where $t\geq 6$, the only best first move is to cell 2.
