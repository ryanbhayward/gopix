\section*{a pruning theorem}
Here, we discuss go assuming no-suicide Tromp-Taylor rules,
so with the positional superko rule.

So, define the {\em history} of a go state as the sequence
$\mP = (P_0, \ldots, P_{t-1})$
of board positions $P_j$, 
where $P_0$ and $P_{t-1}$ 
are respectively the original and current position,
and $|\mP| = t$ is the number of positions in the history.
Define a {\em go state} $S=(\mP,x)$ by its history $\mP$
and by the player to move (ptm) $x$.

Label the cells of the linear go board from 0 to $n-1$
in the obvious way, ie.\ so that,
for $1 \leq j \leq n-1$, cells $j-1$ and $j$ are neighbours.

For a go state $S$, 
$\mu_x(S)$ is the minimax value for $x$ of $S$.

For cells $x,y$ of a position $P$,
the {\em x,y-interchange} $\phi(x,y,P)$ of $P$ is
the position $P'$ obtained from $P$ by 
interchanging the contents of cells $x$ and $y$.
For example,
let $P$ be the linear go position {\tt ( x - o o - ) }.
Then $\phi(0,1,P)$ is the position {\tt ( - x o o - ) }.

Similarly,
the {\em x,y-interchange} $\phi(x,y,\mP)$ of $\mP$
is the history $\mP'$ obtained from $S$ by interchanging
the contents of cells $x$ and $y$ in each position $P_j$ of $\mP$.

\begin{theorem}
Let $S^0 =(\mP,z)$ be a go state in which the current position
$P_j$ has cell 0 $x$ and cell 1 empty.
Let $\mP' = \phi(0,1,\mP)$, and let
$S^1 = (\mP',z)$ be a valid go state, ie.\ the sequence
of positions in $S_1$ corresponds to a sequence of legal moves.
Then $\mu_x(S^0) \leq \mu_x(S^1)$.
\end{theorem}

We will sometimes represent positions pictorially:
eg.\ ({\tt x - }$\rho$) is a position with cell 0 x, cell 1 empty,
and remaining cell sequence $\rho$.

~

\begin{proof}
The current positions of $S^0$ and $S^1$ are respectively
({\tt x - }$\rho$) and ({\tt - x }$\rho$).
We want to find the minimax value for x of $S^0$ and $S^1$,
so let the player-to-move z be o, the opponent of x. 

First assume that $S_0$ is a terminal state:
the last move was pass, as was the immediately previous move by x.
Notice that this implies $S_1$ is also terminal.
Here, the minimax value is just the Tromp-Taylor score:
if the first non-empty cell in $\rho$ is x, $\mu_x(S_0)=\mu_x(S_1)$;
if not, $\mu_x(S_0) = \mu_x(S_1) - 1$.
So the theorem holds.

Next assume that neither $S_0$ nor $S_1$ are terminal.
Let $L^0$ and $L^1$
be the set of legal moves for o from $S^0$ and $S^1$ respectively.
$\mP'$ is the 0,1-interchange of $\mP$,
so $L^0 - \{0,1\} = L^* = L^1 - \{0,1\}$.

From $S^0$,
the move by o to cell 1 captures the x in cell 0,
so 1 is in $L^0$ if and only if the resulting position
({\tt - - }$\rho'$) is not in $\mP$,
where $\rho' = \rho$ unless $\rho$ begins with a sequence
$\sigma$ of consecutive x's followed by either o or the end of the board,
in which case these x's are also captured when o plays at cell 1.

Now consider from $S^1$ the move by o to cell 0.
If $\rho' = \rho$, then $\rho$ does not begin with a sequence $\sigma$
as described above, and this move does not capture the x
at cell 1, so this move is not legal.
Also, if $\rho' \neq \rho$, then
$\rho$ does begin with $\sigma$,
and so the position after o captures the x's at cells \{1\} 
and the locations of $\sigma$ will be exactly the same,
except for interchanging cells 0 and 1,
as the position after o captures the x's by playing at cell 1 from $S^0$.
So, since $\mP'$ is the 0,1-interchange of $\mP$,
if this move by o from $S^1$ to cell 0 is legal,
the so is the move by 1 from $S^0$ to cell 1.

So there are 3 cases:
$L^0 = L^* = L_1$;
$L^0 = L^* \cup \{1\}, L_1 = L*$;
$L^0 = L^* \cup \{1\}, L_1 = L* \cup \{0\}$.

In each case, observe that for every cell $c$ in $L^*$,
the position $Q_0$ obtained from the o-move to c from $S^0$ 
is the 0,1-interchange of 
the position $Q_1$ obtained from the o-move to c from $S^1$.
Also, the hypotheses of the theorem apply to $Q_0$ and $Q_1$,
so by induction we can assume that
$\mu_x(Q_0) \leq \mu_x(Q_1)$.

We argue by induction 
(we need to reformulate the statement of the theorem so
that it holds for any states reached from the original
by a parallel sequence of moves, ie.\ that differ only
by interchanging cells 1 and 2)
on {\tt "} that
$\mu_o(${\tt o-"}$) \leq \mu_o(${\tt -o"}), so
$\mu_x(${\tt -o"}$) \leq \mu_x(${\tt o-"}).
$\mu_x(${\tt x-'}$) \leq \mu_x(${\tt -x'}).
This, and the fact that {\tt -o"} exists (even though {\tt o-"} exists
only if the capture is legal)
implies that $\mu_x(${\tt x-"}$=S_1) \leq \mu_x(${\tt -x"}$=S_2)$.
\end{proof}
\vfill~
