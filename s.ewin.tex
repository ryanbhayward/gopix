\section*{early win detection}
One way to prune moves is with {\bf early win detection:}
recognize the final minimax score before
the final state is reached.
For example, 
sometimes it can be seen by examining a position
that, regardless of the game history
(which can make moves illegal by the superko rule),
white has no legal moves.


One form of win detection uses {\bf Benson-safe positions}.
We will call a linear go position {\bf black Benson-safe}
if from that position, regardless
of game history,
the only legal white move is pass.
\noindent
{\bf observe: ~}

{\bf for any shape empty go board,
the first-player minimax score $t$ is non-negative.}

\noindent
{\bf proof. ~}
use strategy stealing.
argue by contradiction:
assume $t$ negative.
consider game after first move is pass.
now opponent has empty board.
~
case 1: opponent passes, game ends, score 0.
~
case 2: opponent does not pass.
from this point, after exchanging 
the names of the players,
the game tree is identical to the
original game tree,
so opponent's minimax score is $t < 0$,
so opponent prefers case 1, pass, minimax score 0.
~
so, from original position,
first player has a move (pass) with minimax score 0,
which is greater than the minimax score $t$, contradiction.
~ qed.
\vfill

\noindent
{\bf similarly:}

{\bf for any shape empty go board,
let first-player minimax score be $t$.
then first-player minimax score
after first move pass is $-t$.}
~ 
we leave the proof as an exercise.
\vfill

\[\begin{tikzpicture}[x=1cm, y=2cm
  ,every edge/.style={draw, postaction={decorate,decoration={markings,}}}
]
\vertex (a) at (5,10) [label=right:a]{\pic{-----}};
\vertex (b) at (2, 9) [label=right:-a]{ };
\vertex (c) at (4, 9) [label=right:b]{\pic{x----}};
\vertex (d) at (6, 9) [label=right:c]{\pic{-x---}};
\vertex (e) at (8, 9) [label=right:d]{\pic{--x--}};
\vertex (f) at (4, 8) [label=right:$\leq$-c]{\pic{-o---}};
\vertex (g) at (6, 8) {\pic{-x-o-}};
\vertex (h) at (8, 8) {\pic{--xo-}};
\path
(a) edge (b) edge (c) edge (d) edge (e)
(c) edge (f)
(d) edge (g)
(e) edge (h)
;
\end{tikzpicture}\]
\vfill~
